\subsection{latex}

\begin{verbatim}
latex 添加windows字体
wget --http-user=jkxjxxz --http-password=swfcxs http://cs2.swfc.edu.cn/pub/resources/tools/msfonts-chinese.tgz
tar zxf msfonts-chinese.tgz
mv msfonts-cn /usr/share/fonts/truetype/
sudo apt-get install xfonts-wqy ttf-wqy-microhei ttf-wqy-zenhei
fc-cache
sudo fc-cache -f -s -v
fc-list
fc-list :lang=zh
--关于问题 The font "[SIMKAI.TTF]" cannot be found.???
sudo gedit /usr/share/texlive/texmf-dist/tex/latex/ctex/fontset/ctex-xecjk-winfonts.def
fc-list :lang=zh-cn
来查看系统中的中文字体,记下楷体和仿宋对应的名称,即显示信息中第一个英文
在我的系统中楷体是 KaiTi,仿宋是 FangSong
不过会因为安装的字体版本不同而有所差异
接下来只要将对应的字体修改即可,即
把[SIMKAI.TTF]修改为KaiTi
把[SIMFANG.TTF]修改为FangSong
需要注意不止一处

============================================================== 
Latex 清除垃圾文件: 
latexmk -c | latexmk -C
==============================================================
texworks 使用
--shift + tab : 
i + shift + tab -> \item
xg + shift + tab -> \gamma
bth + shift + tab -> \begin{theorem} \end{theorem}
--Ctrl+B  Balance Delimiters       选中块(被括号包含的内容)
============================================================== 
latex 添加windows字体
============================================================== 
wget --http-user=jkxjxxz --http-password=swfcxs http://cs2.swfc.edu.cn/pub/resources/tools/msfonts-chinese.tgz
tar zxf msfonts-chinese.tgz
mv msfonts-cn /usr/share/fonts/truetype/
sudo apt-get install xfonts-wqy ttf-wqy-microhei ttf-wqy-zenhei
fc-cache
sudo fc-cache -f -s -v
fc-list
fc-list :lang=zh
--关于问题 The font "[SIMKAI.TTF]" cannot be found.???
sudo gedit /usr/share/texlive/texmf-dist/tex/latex/ctex/fontset/ctex-xecjk-winfonts.def
fc-list :lang=zh-cn
来查看系统中的中文字体,记下楷体和仿宋对应的名称,即显示信息中第一个英文
在我的系统中楷体是 KaiTi,仿宋是 FangSong
不过会因为安装的字体版本不同而有所差异
接下来只要将对应的字体修改即可,即
把[SIMKAI.TTF]修改为KaiTi
把[SIMFANG.TTF]修改为FangSong
需要注意不止一处
""""""""""""""""""""""""""""""""""""""""""""""""""""""""""""""" 
============================================================== 
Package minted Error: You must have `pygmentize' installed to use this package. 
============================================================== 
在编译时加上参数 --shell-escape
$xelatex --shell-escape xxx.tex
""""""""""""""""""""""""""""""""""""""""""""""""""""""""""""""" 
============================================================== 
命令行 查看包帮助命令 texdoc
============================================================== 
$texdoc graphicx   %查看 graphicx 包的帮助
$texdoc lshort      % 查看 not short latex
""""""""""""""""""""""""""""""""""""""""""""""""""""""""""""""" 
============================================================== 
Ubuntu/Mint下LaTeX宏包安装及更新
============================================================== 
1. 首先要找到默认宏包所在目录,这个Google可以查到,我的系统里有两个

/usr/share/texmf/tex/latex 
/usr/share/texmf-texlive/tex/latex 

2. 如果是安装一个新的宏包,就直接把宏包的压缩文件扔进第一个目录下,直接解压就行,注意解压后的文件里可能有安装说明,照着安装说明做就是了
。如果是更新一个宏包,一般都可以在第二个目录下找到,把原先的宏包重命名成*-backup,再解压新下载的宏包压缩文件,同时如果有安装说明的话,也照着做。 

3. 之后要对宏包重新标记下,终端下执行
# texhash 
4. Log off / Log in后,就完成了
""""""""""""""""""""""""""""""""""""""""""""""""""""""""""""""" 
============================================================== 
安装texlive 2016 找不到 tlmgr 命令
""""""""""""""""""""""""""""""""""""""""""""""""""""""""""""""" 
sudo env PATH="$PATH" tlmgr
============================================================== 
\end{verbatim}